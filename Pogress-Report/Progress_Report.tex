% !TEX TS-program = pdflatex
% !TEX encoding = UTF-8 Unicode

% This is a simple template for a LaTeX document using the "article" class.
% See "book", "report", "letter" for other types of document.

\documentclass[11pt]{article} % use larger type; default would be 10pt

\usepackage[utf8]{inputenc} % set input encoding (not needed with XeLaTeX)


%%% PAGE DIMENSIONS
\usepackage{geometry} % to change the page dimensions
\geometry{a4paper} % or letterpaper (US) or a5paper or....


\usepackage{graphicx} % support the \includegraphics command and options

% \usepackage[parfill]{parskip} % Activate to begin paragraphs with an empty line rather than an indent

%%% PACKAGES
\usepackage{booktabs} % for much better looking tables
\usepackage{array} % for better arrays (eg matrices) in maths
\usepackage{paralist} % very flexible & customisable lists (eg. enumerate/itemize, etc.)
\usepackage{verbatim} % adds environment for commenting out blocks of text & for better verbatim
\usepackage{subfig} % make it possible to include more than one captioned figure/table in a single float
\usepackage{cite}
\usepackage{textcomp}
% These packages are all incorporated in the memoir class to one degree or another...

%%% HEADERS & FOOTERS
\usepackage{fancyhdr} % This should be set AFTER setting up the page geometry
\pagestyle{fancy} % options: empty , plain , fancy
\renewcommand{\headrulewidth}{0pt} % customise the layout...
\lhead{}\chead{}\rhead{}
\lfoot{}\cfoot{\thepage}\rfoot{}

%%% SECTION TITLE APPEARANCE
\usepackage{sectsty}
\allsectionsfont{\sffamily\mdseries\upshape} % (See the fntguide.pdf for font help)
% (This matches ConTeXt defaults)

%%% ToC (table of contents) APPEARANCE
\usepackage[nottoc,notlof,notlot]{tocbibind} % Put the bibliography in the ToC
\usepackage[titles,subfigure]{tocloft} % Alter the style of the Table of Contents
\renewcommand{\cftsecfont}{\rmfamily\mdseries\upshape}
\renewcommand{\cftsecpagefont}{\rmfamily\mdseries\upshape} % No bold!
\newcommand{\textapprox}{\raisebox{0.5ex}{\texttildelow}}



\title{Creating Classifiers for the Identification of Subject Covariates in Brain Graph Data}
\author{Chris Micek, Monica Rodriguez, Addison Wright}
%\date{} % Activate to display a given date or no date (if empty),
         % otherwise the current date is printed 

\begin{document}
\maketitle

\section{Summary}
We have met all goals to-date: We have working classifiers that when presented with both weighted and binary brain graph data of males and females, are able to accurately classify novel brain graphs \textapprox{68}\% of the time. A linear classifier generally works best; however, we have also grouped an ensemble of classifiers in MATLAB that vote on which class data belongs, and this is correct between 64 - 70\% of the time. We have begun feature extraction to determine which network features are most prominent each classifier's decision. We have extracted a particular connection that the linear deemed most important, that Monica will research for further discussion, and will features for the other classifiers by Monday.

\section{Updated Goals}
We wound up implementing more classifiers than we actually expected; curiosity got the better of us, and once we got one to work (a linear classifier) we tested more to see which worked best (and this wound up being the linear classifier most often). We had working code on Wednesday, before our soft Thursday deadline, and are on track to complete the goals listed in our proposal. We still have to cross-validate our classifiers, which we will do this weekend, refining our program and hopefully applying it to non-binary covariates.

\section{Updated Timeline}
Our timeline remains unchanged; we will have refined code by the end of the weekend and a complete poster by next Friday.

\bibliography{yourbibname}{}
\bibliographystyle{plain}

\end{document}








